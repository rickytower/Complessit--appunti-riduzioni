\section{Riduzioni}
\begin{definition}[Riduzione polinomiale tra problemi (alla Karp)]
	Un problema $\pbc{A}$ si riduce polinomialmente (alla Karp) a $\pbc{B}$ e si scrive $\pbc{A}\preceq\pbc{B}$ se $\exists$ un algoritmo polinomiale $f$ tale che \[\forall x \in \pbc{A}\tc \pbc{A}(x)=\yes \iff \pbc{B}(f(x))=\yes\] 
	L'effetto di $f$ è
	\[\underbracket{|x|}_{\substack{\textit{istanza}\\\textit{di $\pbc{A}$}}}\xlongrightarrow{\textit{effetto di }f}\underbracket{|y|}_{\substack{\textit{istanza}\\\textit{di $\pbc{B}$}}}\in O(|x|^c)\ive y=f(x)\]
	in tempo $O(|y|^d)$ si risponde \[\boxed{\pbc{B}(y)=\yes\iff\pbc{A}(x)=\yes}\] in tempo $O(|x|^c)+O(|y|^d)$, che è polinomiale.
\end{definition}
\subsection{Riduzione k-col a k+1-col}
Un grafo $G$ è propriamente colorato se \[\forall (u,v)\in E(G)\tc u\ne v \ive colore(v)\ne colore(u)\]
\begin{demonstration}
	\begin{figure}[thbp]
		\centering
		\includegraphics{k-coloring/k-coloring.pdf}
		\caption{riduzione k-coloring a k+1-coloring.}
		\label{fig:k-coloring}
	\end{figure}
	Fornire la funzione polytime $f$ tale che se $\forall G$ è $k-col$ allora \[G\ \textit{è k-colorabile}\iff f(G)\ \textit{è k+1-colorabile}\]
	
	$f$ è \textit{polytime computable}. Se $G$ è $k-col$ allora $G'$ è $k+1-col$. 
	\begin{itemize}
		\item quando viene colorato $G'$ per i vertici che erano in $G$ si tiene la k-colorazio-ne che era presente e per $v^\ast$ viene utilizzato il colore in più; 
		\item se $G'$ è k+1-col \textit{wlog\footnote{without loss of generality.}} si imposta a $k+1$ il colore in $v^\ast$. Poiché $\forall v\ne v^\ast$ $(v,v^\ast)\in E(G')$ si ha che $colore(v)\ne k+1$ e $colore(v)\in\{1,\dots,k\}$
		\item se $\forall v,w\in V(G)\tc e=(v,w) \ive colore(v)\ne colore(w)$ allora la colorazione è propria per $G$.
	\end{itemize}
	Per dimostrare l'implicazione inversa, basta partire da $G'$ e mostrare che il risolutore restituisce \textit{yes} per $k+1-col$, togliere $v^\ast$ e gli archi che lo congiungono, togliere il $k+1$ colore e mostrare che anche il risolutore di $k-col$ restituisce \textit{yes}.
	\qed
\end{demonstration}

Da $k-col\preceq k+1-col$ segue che
\begin{gather*}
	\exists f\tc G\to G'=f(G)\ive |G'|=|G|^c_f\quad [polytime]
\end{gather*}
Se esiste $A$ per $k+1-col$ $A(G)=k+1-col$ e $T_A(G')=O(|G'|^c_a)\implies B(G)=A(f(G))$ $B$ è un algoritmo \textit{polytime} per $k-col$. Il tempo di $B$ su $G$ è
\[T_B(G)=O(|G|^c_d)+T_A(f(G))=O(|G|^c_d)+O((|G|^c_d)^c_A)=O(|G|^{c_d\cdot c_A})\]

\subsubsection{Alcune osservazioni}
\begin{gather*}
	\pbc{A}\preceq\pbc{B} \ive \pbc{B}\in\p\implies \pbc{A}\in\p\\
	\pbc{A}\preceq\pbc{B}\ive \pbc{A}\notin\p\implies \pbc{B}\notin\p
\end{gather*}

Se $k-col\notin\p$ allora $B$ non può esistere. Questo implica che $A$ non può esistere e $k+1-col\notin\p$.

Si supponga che esista $\pbc{B}$ tale che \[\forall \pbc{A}\in \np\quad\pbc{A}\preceq\pbc{B}\]
\begin{figure}[thbp]
	\centering
	\includegraphics{p-np-npc/p-np-npc.pdf}
	\caption{relazioni tra le classi di complessità.}
	\label{fig:p-np-npc}
\end{figure}
\begin{itemize}
	\item Se $\pbc{B}\in\p$ allora $\pbc{A}\in\p$. Segue che $\forall \pbc{A}\in \np$ risulta $\np\subseteq \p\implies \p=\np$
	\item Se $\p\ne \np\implies B\notin\p$
\end{itemize}
\begin{definition}[NP-completezza]
	Si dice che $\pbc{B}$ è NP-completo ($\npc$) se 
	\begin{itemize}
		\item $\pbc{B}\in \np$
		\item $\forall \pbc{A}\in \np\tc \pbc{A}\preceq\pbc{B}$ ovvero $\pbc{B}$ è NP-hard
	\end{itemize}
\end{definition}
\subsection{Problemi NPC}
Esistono problemi NP-completi. Se $\pbc{A}\in\np$ se esiste un verificatore $B(\cdot,\cdot)$ ploytime per $\pbc{A}$ tale che \[\forall x \in \ipb{A}\tc\pbc{A}(x)=\yes\iff\exists y\tc B(x,y)=\yes\]
\subsection{Problema SAT (Satisfiability)}
\problem{SAT}{Formula CNF $\phi$}{$\yes\iff\phi$ è soddisfacibile}
\begin{definition}[Forumla CNF]
	Una formula è CNF\footnote{Conjunctive Normal Form.} se \[\phi(x_1\dots x_n)=\underbracket{C^1\ive C^2\ive\dots \ive C^n}_\textit{congiunzione di clausole}\] dove ogni $C^i\tc 1\le i\le n\ive l^i_j\tc 1\le j\le k$ \[C^i=\underbracket{l_1^i\ve l_2^i\ve\dots\ve l_k^i}_\textit{disgiunzione di letterali}\qquad l_j^i\in\{\underbracket{x_1,\dots,x_n,\overline{x_1},\dots,\overline{x_n}}_\textit{insieme di variabili\footnote{variabili true e false.}}\}\]
\end{definition}
\begin{definition}[Assegnamento]
	Un assegnamento $a$ è definito come\[a=(a_1,\dots,a_n)\in\{T,F\}^n\]
	Si dice che un assegnamento $a$ soddisfa $\phi$ se $\boxed{\phi(a_1,\dots,a_n)=T}$
\end{definition}
\subsection{Riduzione k-col a SAT}
Dato $G$ è possibile costruire in \textit{polytime} $\phi_G$ CNF tale che $G$ è k-colorabile se e solo se $\phi_G$ è soddisfacibile.

\paragraph{Dimensioni delle istanze}La taglia di G è\[|G|=|(V,E)|=|V|+|E|\] mentre quella di $\phi$ è \[|\phi|=n\] con $n$ il numero di letterali che sono contenuti all'interno di $\phi$.

\paragraph{Obbiettivo}L'obbiettivo che si vuole raggiungere è definire una funzione $f$ \textit{polytime computable} che permetta di ridurre le istanze del problema $k-col$ a istanze del problema SAT. In altre parole, dare una definizione di $k-col$ ridefinendo le sue istanze in maniera tale da utilizzare le istanze di SAT. Si definisce $\phi_G$ come 
\begin{equation}
	\boxed{\phi_G=\bigwedge_{v\in V}(C^v\ive D^v)\ive \bigwedge_{e\in E}E^e}\label{eqn:f-k-col-sat}
\end{equation}
tale che
\begin{align}
	\forall v\in V \tc&\begin{cases}
		C^v =x_1^v\ve x_2^v\ve\dots\ve x_k^v\\ D^v=\bigwedge_{1\le i \le j\le k}(\overline{x_i^v}\ve \overline{x_j^v})
	\end{cases}\label{eqn:sistema-vertici-kcol-sat}\\
	\forall e=(u,v)\in E\tc& E^e=\bigwedge_{1\le i\le k}(\overline{x_i^u}\ve \overline{x_i^v})\label{eqn:archi-kcol-sat}
\end{align}
dove \cref{eqn:sistema-vertici-kcol-sat} contiene due condizioni:
\begin{enumerate}
	\item ciascun vertice deve avere almeno un colore
	\item ciascun vertice non può contenere più di un colore
\end{enumerate}
che si traduce in ``\textit{ogni vertice deve avere esattamente un colore}''. La \cref{eqn:archi-kcol-sat} garantisce la colorazione propria del grafo. La definizione scelta permette di specificare in posizione apice di una variabile il vertice, e in posizione pedice il colore. La dimensione di $|\phi_G|$ diventa \[|\phi_G|=|V|\left(k+\binom{k}{2}2\right)+2k|E|\]
ovvero ciascun vertice dell'insieme $V$ ha una clausola $C^v$ lunga $k$ letterali, una clausola $D^v$ lunga 
per ogni coppia scelta di colori $\binom{k}{2}$ si hanno due letterali $\overline{x_i^v}$ e $\overline{x_j^v}$ e ciascun arco nell'insieme $E$ ha una clausola $E^e$ lunga 
per ogni possibile colore (i colori sono $k$), due letterali.
\subsubsection{Esempio}
Si estraggono le informazioni dal grafo
\begin{figure}[thbp]
	\centering
	\includegraphics[width=0.3\linewidth]{k-col-sat/k-col-sat-graph-example}
	\caption{dal grafo si estraggono le informazioni per ricavare la formula CNF.}
	\label{fig:k-col-sat-graph-example}
\end{figure}
e si ricavano le equazioni
\begin{align*}
	C^a&=x_1^a\ve x_2^a \qquad  C^b=x_1^b\ve x_2^b \qquad C^c=x_1^c\ve x_2^c\\
	D^a&=\overline{x_1^a}\ve \overline{x_2^a} \qquad D^b = \overline{x_1^b}\ve\overline{x_2^b} \qquad D^c=\overline{x_1^c}\ve\overline{x_2^c}\\
	E^{e_1}&=(\overline{x_1^a}\ve\overline{x_1^b})\ive(\overline{x_2^a}\ve\overline{x_2^b})\\E^{e_2}&=(\overline{x_1^c}\ve\overline{x_1^b})\ive(\overline{x_2^c}\ve\overline{x_2^b})\\E^{e_3}&=(\overline{x_1^a}\ve\overline{x_1^c})\ive(\overline{x_2^a}\ve\overline{x_2^c})
\end{align*}
In questo caso i colori a disposizione sono 1 e 2 (pedice delle variabili). Il grafo non ha colorazione propria perché non si riesce a trovare una combinazione di colori sui vertici tale per cui \cref{eqn:archi-kcol-sat} è vera.
Si dimostra la riduzione.
\begin{demonstration}[$\implies$]
	Si assume $G$ k-colorabile. Sia $\{c(v)\tc v\in V\}$ una colorazione propria ovvero $\forall e=(u,v)\in E\tc c(u)\ne c(v)$ e con $c(v)\in\{1,\dots,k\}$. Si definisce l'assegnamento $a=(a^{v_1}_1,\dots, a^{v_1}_k,\dots, a^{v_n}_1,\dots,a^{v_n}_k)$ e si assegna 
	\begin{equation}
		a^{v_i}_j=\begin{cases}
			T & c(v_i)=j\\
			F & c(v_i)\ne j
		\end{cases}\label{eqn:assign-definition}
	\end{equation}
	Si mostra quindi che ogni gruppo di clausole è soddisfatto dall'assegnamento. Preso un $C^{|v|}(a)$\footnote{abuso di notazione, si immagini ci siano scritte tutte le variabili dell'assegnamento $a$.} si ha che l'assegnamento rende vera ogni clausola della \cref{eqn:f-k-col-sat}
	\begin{itemize}
		\item $C^v(a)=T$ perché per ogni vertice, in particolare per $v$, una variabile associata ha valore $T$ (esiste un legame tra il vertice e il colore per definizione di $a^{v_i}_j$);
		\item $D^v(a)=T$ perché nella definizione viene associato univocamente un valore alla variabile;
		\item $E^v(a)=T$ perché la colorazione è propria (poiché viene assunto che $G$ sia $k-col$ e che la colorazione associata a $G$ sia propria), ovvero $\forall e=(u,v)\tc c(u)\ne c(v)$. Se fosse $E^v(a)=F$ allora \[\exists i\tc \overline{x_i^u}\ve \overline{x_i^v}=F\implies x_i^u=T\ive x_i^v=T\implies a_i^u=T\ive a_i^v=T\]
		ma per la \cref{eqn:assign-definition} risulta $c(u)=i \ive c(v)=i \implies c(u)=c(v)$ in contraddizione con l'assunzione che la colorazione è propria.
	\end{itemize}
	concludendo che $\phi_G(a)=T$.\qed
\end{demonstration}
Per dimostrare la coimplicazione si parte da un assegnamento che rende vera una formula $\phi$ senza conoscere il grafo.
\begin{demonstration}[$\impliedby$]
	Si assume che $\phi_G(a)=T$ e si mostra che $G$ è $k-col$. L'assegnamento $a=(a_1^{v_1},\dots,a_n^{v_1},\dots,a_1^{v_n},\dots,a_n^{v_n})$ tale che $\phi_G(a)=T$. Si definisce una colorazione per $G$ basata su $a$:
	\[\forall v\in V\tc c(v)=i\iff a_i^v=T\]
	\begin{enumerate}
		\item ogni vertice ha un solo colore, infatti se $\phi(a)=T$ segue che:
		\begin{itemize}
			\item $C^v(a)=T\implies\exists i \tc a_i^v=T \implies c(v)=i$
			\item $D^v(a)=T\implies \notexists i,j\tc a_i^v=T\ive a_j^v=T$
		\end{itemize}
		da cui \[\onlyexists i\tc c(v)=i\]
		\item ogni arco \textbf{non è monocromatico} ovvero $c(u)\ne c(v)$. Poiché $\phi(a)=T\implies E^e(a)=T\implies\notexists  i\tc a_i^u=T \ive a_i^v=T\implies c(u)\ne c(v)$.
	\end{enumerate}
	Da cui $G$ è $k-col$. \qed
\end{demonstration}
Dunque $k-col\preceq SAT$.
\begin{figure}[thbp]
	\centering
	\includegraphics{k-col-sat/sat-is-npc}
	\caption{tutti i problemi k-col si riducono a SAT. Si vedrà che tutti i problemi si riducono a SAT.}
	\label{fig:sat-is-npc}
\end{figure}
\subsection{Problema Circuit-SAT}
\begin{figure}[thbp]
	\centering
	\includegraphics[]{circuit-sat/circuit-sat.pdf}
	\caption{circuito booleano.}
	\label{fig:circuit-sat}
\end{figure}
\problem{Circuit-SAT}{Circuito booleano $C$}{$\yes\iff C$ è soddisfacibile}
\begin{definition}[Circuito booleano]
	È un grafo aciclico diretto in cui ogni vertice ha un in-degree=1,2 tranne i vertici di input che hanno un in-degree=0.
	
	Ogni vertice ha un out-degree=1 e ha associato un \textbf{operatore booleano} \textit{and}, \textit{or}, \textit{not}. In particolare \textit{not} è associato a tutti i vertici con in-degree=1 mentre \textit{and} e \textit{or} a quelli con in-degree=2.
\end{definition}
Dato un assegnamento in input a $x_1,\dots,x_n$ allora $C(x_1,\dots,x_n)=v$ con $v$ valore dell'arco in output. Nell'esempio in \cref{fig:circuit-sat} si ha $C(0,1,1,1,1,1)=0$.
\subsection{Teorema di Cook-Levin}
\begin{theorem}[Cook-Levin]
	SAT è NP-completo\label{thm:cook-levin}
\end{theorem}
Per dimostrare il \cref{thm:cook-levin} si deve dimostrare che ($\color{Green}\checkmark$ = già dimostrato)
\begin{itemize}
	\item $\text{SAT}\in \np$ $\color{Green}\checkmark$
	\item $\text{Circuit-SAT}\preceq \text{SAT}$
	\item Circuit-SAT è NP-completo
	\begin{itemize}
		\item $\text{Circuit-SAT}\in\np$ $\color{Green}\checkmark$
		\item Circuit-SAT è NP-hard \[\forall\pbc{A}\in\np\tc\pbc{A}\preceq \text{Circuit-SAT}\]
	\end{itemize}
\end{itemize}
\begin{lemma}
	Per ogni problema $\pbc{B}\in\p$ e per ogni $n\in\n$\[\exists C_n\tc\forall x\in \ipb{B}\quad |x|=n \ive C_n(x)=\pbc{B}(x)\]
	Inoltre $C_n$ è computabile in tempo \textit{polytime} ovvero $\exists c\tc O(|x|^c)$.\label{lemma:circuito-booleano-algo}
\end{lemma}
Il $\cref{lemma:circuito-booleano-algo}$ afferma che, fissato un input $x$ con $|x|=n$, un algoritmo $A$ può essere tradotto in un circuito booleano $C_n$ tale che \[A(x)=C_n(x)\]
\begin{demonstration}[Circuit-SAT è NP-hard]
	Affermare \[\forall\pbc{A}\in\np\tc\pbc{A}\preceq\text{Circuit-SAT}\]
	significa \[x\in\ipb{A}\xrightarrow[\textit{in}\ x]{\polytime}C_x^\pbc{A}\tc\pbc{A}(x)=\yes\iff C_x^\pbc{A}\ \textit{è soddisfacibile}\]
	Sostenere che $\pbc{A}\in\np$ implica che
	\[\exists B(x,y)\tc\forall x\in\ipb{A}\quad\ipb{A}(x)=\yes\iff\exists y\in\{0,1\}^{|x|^c}\]
	
	Fissato $x$, il problema associato al calcolo di $B(x,y)$ 
	\problem{Calcolo di $\mathbf{B(x,y)}$}{$y$}{$B(x,y)$}
	è in $\p$. Seguendo quanto riportato dal \cref{lemma:circuito-booleano-algo} si può affermare che \[\exists C_n^\pbc{A}\tc \forall y\quad |y|=n \quad C_n^\pbc{A}(y)=B(x,y)\] $C_n^\pbc{A}$ è calcolabile in polytime nell'input, quindi $O(n^d)=O(|y|^d)=O(|x|^{c\cdot d})$.
	\begin{align*}
		{\color{Red}x\in\ipb{A}\textit{ è yes }}&\iff \exists y \tc B(x,y)=\yes\\&\iff \exists y \tc C_n^\pbc{A}(y)=\yes\\&\iff C_n^\pbc{A}\ \textit{è un'istanza yes per Circuit-SAT}
		\\&\iff \color{Red}\textit{Circuit-SAT}(C_n^\pbc{A})=\yes
	\end{align*}
	da cui si ottiene \[x\in\ipb{A}\textit{ è yes }\iff \textit{Circuit-SAT}(C_n^\pbc{A})=\yes\]
	Quindi $\textit{Circuit-SAT}\preceq \textit{SAT}$\qed
\end{demonstration}

Si dimostra che Circuit-SAT $\preceq$ SAT facendo riferimento ad un circuito booleano arbitrario in \cref{fig:dim-circuitsat-to-sat}
\begin{figure}[thbp]
	\centering
	\includegraphics{circuit-sat/circuitsat-to-sat}
	\caption{circuito booleano della dimostrazione.}
	\label{fig:dim-circuitsat-to-sat}
\end{figure}
\begin{demonstration}[Circuit-SAT $\preceq$ SAT]
	\[\exists x \tc C(x)=T\iff \exists x'\tc \phi(x')=T\] Si pone $x_1\ve x_2=y_1$, $\overline{x_3}=y_2$ e $y_3=y_1\ive y_2$ da cui si ottiene la formula $\phi$ \[\phi(x_1,x_2,x_3,y_1,y_2,y_3)=[(x_1\ve x_2)=y_1]\ive[\overline{x_3}=y_2]\ive[y_3=y_1\ive y_2]\ive y_3\] 
	\paragraph{Osservazione} $a=b\equiv (\overline{a}\ve b)\ive(a\ve \overline{b})$
	
	Quindi \[\exists x \tc C(x)=T\iff\exists x,y\tc \phi(x,y)=T\]
	Si sa che $\forall\pbc{A}\in N\p\quad \pbc{A}\preceq\text{Circuit-SAT}\preceq\text{SAT}$
	\[x\rightarrow C_x^\pbc{A}\rightarrow\phi\]
	e dunque $\pbc{A}\preceq \text{SAT}$. SAT è NP-hard e appartiene alla classe $\np$. Dunque SAT è NP-completo.\qed
\end{demonstration}
\subsubsection{Dimostrare che un problema è NP-completo}
Per dimostrare che un problema $\pbc{D}$ è NP-completo basta mostrare che
\begin{itemize}
	\item $\pbc{D\in\np}$ facile perché basta mostrare che una soluzione è verificabile in tempo polinomiale;
	\item $\forall \pbc{A}\in\np\quad \pbc{A}\preceq\pbc{D}$ la si dimostra usando la transitività delle riduzioni. Si sceglie un problema $\pbc{B}\in\npc$ e si dimostra che $\pbc{B}\preceq\pbc{D}$. Poiché $\forall \pbc{A}\quad \pbc{A}\preceq\pbc{B} \ive \pbc{B}\in\npc\implies\forall \pbc{A}\quad \pbc{A}\preceq\pbc{D}$ (NP-hardness).
\end{itemize}	

\begin{definition}[k-CNF]
	Una formula k-CNF è una CNF in cui tutte le clausole hanno al più $k$ letterali.
\end{definition}
Si consideri la formula CNF
\begin{equation}
	\phi=\underbracket{(x_1\ve x_2\ve x_3)}_{C^1}\ive\underbracket{(x_1\ve\overline{x_2}\ve\overline{x_4}\ve x_5)}_{C^2}\ive \underbracket{(x_1\ve\overline{x_3}\ve \overline{x_4}\ve x_5\ve x_6)}_{C^3}\label{eqn:esempio-3-cnf}
\end{equation}
\problem{3-SAT}{3-CNF $\phi$}{$\yes\iff \phi$ è soddisfacibile}
\paragraph{Osservazione} $\text{2-SAT}\in\np$.
$\text{3-SAT}\in\np$ come per SAT. Si vuole dimostrare che 3-SAT è NP-completo, mostrando che 
\begin{align*}
	\text{SAT}&\preceq\text{3-SAT}\\
	\underbracket{\phi}_{\text{CNF}}&\rightarrow\underbracket{\phi'}_{\text{3-CNF}}
\end{align*}
Data una clausola $C$ che è formata da più di 3 letterali, si costruiscono le clausole $D_1,D_2,\dots,D_t$ con 3 letterali tale che $C$ è soddisfacibile $\iff D_1\ive D_2\ive\dots\ive D_t$ è soddisfacibile. Si supponga che \[C=l_1\ve l_2 \ve l_3\ve l_4\ve \dots\ve l_k\quad \textit{con}\quad k>3\]
dove $l_i\in\{x_1,\overline{x_1},\dots,x_n,\overline{x_n}\}$. La clausola CNF si può riscrivere come una 3-CNF
\[(l_1\ve l_2\ve z_1)\ive(\overline{z_1}\ve l_3\ve z_2)\ive (\overline{z_2}\ve l_4\ve z_3)\ive \dots\ive (\overline{z_{k-3}}\ve l_{k-1}\ve l_k)\]
e considerando la $\phi$ e passandola in forma 3-CNF diventa
\[C'=(l_1\ve l_2\ve z_1)\ive(\overline{z_1}\ve l_3\ve z_2)\ive (\overline{z_2}\ve l_4\ve z_3)\ive (\overline{z_3}\ve l_5\ve l_6)\]
che contiene clausole di taglia 3. 
Alcune osservazioni:
\begin{itemize}
	\item il numero dei letterali della nuova clausola $C'$ è meno del doppio di quella originale $C$;
	\item se $C$ è soddisfacibile, esiste un assegnamento che rende soddisfacibile anche la clausola $C'$. Quindi esiste un assegnamento che rende almeno un letterale vero. Usando la scelta che soddisfa $C$ che la rende vera, allora si usa la stessa scelta su $C'$ e conseguentemente deve risultare vera. 
\end{itemize}
Quindi per verificare che $C'$ è soddisfacibile se $C$ è soddisfacibile, si mostra che, dato un letterale che rende vera la clausola, indipendentemente dagli altri letterali, si pongono i valori di $z$ delle clausole vicine ad un valore che le rende vere. Questo genera un effetto ``a catena'' per cui si propaga la scelta a tutte le clausole tale che ciascuna risulterà vera.

Per mostrare la coimplicazione, si suppone che esista un assegnamento
\begin{gather*}
	\exists a_x,a_z\tc  ((l_1\ive l_2\ive z_1)=T \implies C'=T) \ive (a_x\textit{ rende vera } C)
\end{gather*}

Si suppone per assurdo che $a_x, a_z$ soddisfa $C'$ ma tutti gli $l_i$ sono falsi. Se $a_z$ rende $C'$ vero ma $a_x$ rende $C = F$, vuol dire che tutti gli $l_i$ sono falsi. Il problema è che se non è presente almeno un letterale vero, i soli $z$ non sono sufficienti a rendere vera $C'$. Per cui si arriva all'assurdo. Con il verde si indica il vero e con il rosso il falso.
\begin{gather*}
	(\red{l_1}\ve \red{l_2}\ve \green{z_1})\ive(\red{\overline{z_1}}\ve \red{l_3}\ve \green{z_2})\ive (\red{\overline{z_2}}\ve \red{l_4}\ve \green{z_3})\ive \dots\ive (\red{\overline{z_{k-3}}}\ve \red{l_{k-1}}\ve \red{l_k})
\end{gather*}

Tornando a considerare l'\cref{eqn:esempio-3-cnf} la formula $\phi'$ diventa
\begin{gather*}
	\phi'=(x_1\ve x_2\ve x_3)\ive(x_1\ve\overline{x_2}\ve z_1)\ive \underbracket{(\overline{z_1}\ve \overline{x_1}\ve x_5)}_{C^2}\ive\\
	\underbracket{(x_4\ve\overline{x_3}\ve z_2)\ive(\overline{z_2}\ive \overline{x_4}\ve z_3)\ive (\overline{z_3}\ve x_5\ve x_6)}_{C^3}
\end{gather*}


Anche questa versione di 3-SAT è NP-completa.
\problem{3-SAT (esattamente 3)}{una 3-CNF in cui tutte le clausole hanno taglia 3}{$\yes\iff\phi$ è soddisfacibile}
\begin{demonstration}[3-SAT esattamente 3 è NP-completo]
	Data una CNF $\phi$ con clausole da $\le 3$ letterali, possiamo creare una CNF $\varphi$ con clausole da 3 letterali tali che $\phi$ è soddisfacibile se e solo se $\varphi$ è soddisfacibile.
	
	\begin{gather*}
		\phi=C^1\ive C^2\ive\dots\ive C^n\\
		\begin{multlined}
			|C^1|=1\quad C^1=l\implies(l\ve z_1\ve z_2)\ive(l\ve\overline{z_1}\ve z_2)\ive(l\ve z_1\ve \overline{z_2})\ive\\(l\ve \overline{z_1}\ve\overline{z_2})\\
			|C^2=2|\quad C^2=l_1\ve l_2\implies (l_1\ve l_2\ve z_1)\ive (l_1\ve l_2\ve \overline{z_1})
		\end{multlined}
	\end{gather*}
\end{demonstration}
\subsection{NAE-k-SAT (Not all equal k-SAT)}
\problem{NAE-k-SAT (Not all equal k-SAT)}{formula k-cnf $\phi$}{$\yes\iff\phi$ NAE-soddisfa k-SAT (esiste un assegnamento $a$ tale che in ogni clausola $C^i$, $a$ pone almeno un letterale a $T$ e almeno uno a $F$)}
Si consideri l'assegnamento $\phi$
\begin{gather*}
	\phi(x_1,x_2,x_3)=(x_1\ve \overline{x_2}\ve x_3)\ive(\overline{x_1}\ve\overline{x_2}\ve\overline{x_3})
\end{gather*}
con $a=(x_1=T,x_2=F,x_3=T)$. Tale assegnamento non è NAE-soddisfacibile per $\phi$. L'assegnamento $b=(x_1=F,x_2=F,x_3=T)$ NAE-soddisfa $\phi$.
\paragraph{Osservazione} Data $\phi$, se $a=(a_1,\dots,a_n)$ NAE-soddisfa $\phi$ allora anche $\overline{a}=(\overline{a_1},\dots,\overline{a_n})$ NAE-soddisfa $\phi$. Inoltre \begin{equation*}
	\overline{a_i}=\begin{cases}
		F & a_i=T\\ T & a_i=F
		\end{cases}
\end{equation*}
NAE-3-SAT è NP-completo
\begin{enumerate}
	\item $\text{NAE-3-SAT}\in \np$: esiste un verificatore polinomiale $\color{Green}\checkmark$
	\item NAE-3-SAT è NP-hard: $\text{3-SAT}\preceq\text{NAE-4-SAT}\preceq\text{NAE-3-SAT}$. Si dà la dimostrazione in due parti.
\end{enumerate}
\begin{definition}[Riduzione polinomiale]
	$\pbc{A}$ è NP-completo e $\pbc{B}\in\np$ e \[\underbracket{\pbc{A}\preceq\pbc{B}\implies \pbc{B}\ \text{è NP-completo}}_{\text{$\pbc{B}$ è NP-hard}}\]
	
	Una riduzione da $\pbc{A}$ a $\pbc{B}$ ($\pbc{A}\preceq\pbc{B}$) permette di risolvere $\pbc{A}$ in \textit{polytime} se esiste un programma/algoritmo che risolve $\pbc{B}$ in \textit{polytime}.
\end{definition}
Si definisce un risolutore per $\pbc{A}(x)$
\begin{lstlisting}[mathescape, escapeinside=||,frame=single]
	|\textbf{Input: }|$x\in \ipb{A}$
	$y\gets Riduzione(x)$
	$\textbf{return}\ Solutore-\pbc{B}(y)$
\end{lstlisting}
e la subroutine di riduzione
\begin{lstlisting}[mathescape, escapeinside=||,frame=single]
	$Subroutine\ riduzione(x)$
	|\textbf{Input:}| $x\in \ipb{A}$
	|\textbf{Output:}| $y\in\ipb{B}$
\end{lstlisting}
\begin{figure}[thbp]
	\centering
	\includegraphics[width=0.7\linewidth]{dim-nae-3-sat-npc/riduzione.pdf}
	\caption{}
	\label{fig:}
\end{figure}

\begin{demonstration}[$\text{3-SAT}\preceq\text{NAE-4-SAT}$]
	Per mostrare 
\end{demonstration}
